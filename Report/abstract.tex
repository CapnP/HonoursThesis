%%
%% Template abstract.tex
%%

\chapter*{Abstract}
\label{cha:abstract}
\addcontentsline{toc}{chapter}{Abstract}

Term Indexing is an optimisation technique for logical theorem provers which
restricts the search space when attempting to find clauses which fulfill the conditions
of inference rules. This is done by allowing the rapid retrieval of all terms which
are likely to \emph{unify} with a query term. This paper describes in detail
the process of implementing a particular Term Indexing technique, Fingerprint
Indexing, for a new theorem prover known as \beagle. This includes all initial investigations
for how and where indexing could be applied, building the index itself and finally applying the
index to \beagle's inference rules.
Fingerprint Indexing was applied to three of \beagle's inference rules: superposition (the most costly and most often used rule),
demodulation and negative unit simplification. Two distinct versions of indexing were
created, the standard implementation of Fingerprint Indexing and an \emph{enhanced}
version which applies some original optimisations.
These two implementations of \beagle\ were tested alongside unindexed \beagle\ 
against a set of 50 relevant problems from the TPTP library. The standard version
of indexing was able to significantly increase performance for large problems well-suited to indexing;
solving in under a minute a couple of problems which unindexed \beagle\ could not solve at all.
Remarkably, the enhanced version remained useful even in simple problems where indexing would
not normally be worthwhile.
For each superposition inference the final enhanced implementation of Fingerprint Indexing exhibited a
35\% runtime improvement on average; and up to 50\% for large problems.
Each negative unit simplification had its runtime decreased by 94\% on average,
but since the rule is not applied as often as superposition this did not contribute as
significantly to overall runtime. Some tests were also performed to compare
a variety of Fingerprint Indexing configurations. Results from these tests
matched those seen in the paper first propsing Fingerprint Indexing; cross-validitating
both results and solidfying Fingerprint Indexing as a major competitor
in the Term Indexing field.


%%% Local Variables: 
%%% mode: latex
%%% TeX-master: "thesis"
%%% End: 
