%%
%% Template chap1.tex
%%

\chapter{Background}
\label{cha:background}

\section{Sentential and Predicate (First-Order) Logic}
\label{sec:senfol}

\section{Theorem Proving}
\label{sec:proving}

\section{Term Indexing}
\label{sec:indexing}

\section{Fingerprint Indexing}
\label{sec:fingerprint}

\section{The Beagle Theorem Prover}
\label{sec:beagle}

\section{Scala}
\label{sec:scala}

We add a few blank pages here to illustrate section
headings\index{Section~headings} on odd
pages (other than the first page in a new chapter).

Also, here is an example of a citation~\cite{lamport94}, and another
one~\cite{knuth86}, and another~\cite{goossens94}.  If in the context,
it makes sense to talk about the work of an author in a more
integrated way, like \citeN{lamport94}, that can be done
too\index{Books~to~read|textbf}.  Because the information on this page
is \emph{so} important, it has been indexed too (see the index at the
back of this thesis).\footnote{However, it is \emph{strongly}
  advisable to leave indexing until the thesis is complete.  Donald
  Knuth says to allow about a day for the task of indexing.  In my
  experience he was spot-on.}

\clearpage Note the chapter name at the top of this even (lefthand)
page\dots

\cleardoublepage
Note the section heading at the top of this odd (righthand) page\dots

%%% Local Variables: 
%%% mode: latex
%%% TeX-master: "thesis"
%%% End: 
