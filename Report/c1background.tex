%%
%% Template chap1.tex
%%

\chapter{Background}
\label{cha:background}

\section{First-Order Logic and Notation}
\label{sec:senfol}

As stated, \beagle is a \emph{first-order logic} theorem prover. Thus in order to
understand its purpose and functions a basic understanding of this logical system
is required. This section provides a rudimentary overview of FOL syntax and uses;
but also includes an explanation of any specialised notation used throughout the paper.

\begin{itemize}
\item Variables
\item Symbols
\item Predicates
\item Quantifiers
\end{itemize}


\begin{itemize}
\item Positions
\end{itemize}

\section{Automated Reasoning and Theorem Proving}
\label{sec:proving}

Automated Reasoning is a rapidly growing field of research where computer programs
are used to solve problems stated in first order logic statments or other formal logics. 


\section{Term Indexing}
\label{sec:indexing}

\section{Fingerprint Indexing}
\label{sec:fingerprint}

Fingerprint 

\section{The Beagle Theorem Prover}
\label{sec:beagle}

\subsection{The Weak Abstraction with Heirachic Superposition Calculus}

\section{Scala}
\label{sec:scala}

As mentioned above \beagle is written in \emph{Scala}, the Scalable Language. Scala
is a functional language and may be confusing to those who are not familiar with the
functional programming paradigm. This thesis will contain occasional snippets of
Scala code; but note that any snippets used will be accompanied by an explanation
and in general an understanding of Scala is not required.

\begin{listing}[H]
\begin{minted}{scala}
{case ing => foreach do 1 "lol"}
   case
   Case
   new Douglas
          space
             more case space
\end{minted}
\caption{Example of a listing.}
\label{lst:example}
\end{listing}




We add a few blank pages here to illustrate section
headings\index{Section~headings} on odd
pages (other than the first page in a new chapter).

Also, here is an example of a citation~\cite{lamport94}, and another
one~\cite{knuth86}, and another~\cite{goossens94}.  If in the context,
it makes sense to talk about the work of an author in a more
integrated way, like \citeN{lamport94}, that can be done
too\index{Books~to~read|textbf}.  Because the information on this page
is \emph{so} important, it has been indexed too (see the index at the
back of this thesis).\footnote{However, it is \emph{strongly}
  advisable to leave indexing until the thesis is complete.  Donald
  Knuth says to allow about a day for the task of indexing.  In my
  experience he was spot-on.}

%%% Local Variables: 
%%% mode: latex
%%% TeX-master: "thesis"
%%% End: 
