%%
%% Template chap1.tex
%%

\chapter{Background}
\label{cha:background}

\section{First-Order Logic Terms and Notation}
\label{sec:senfol}

This thesis focuses around the extension of \beagle, a \emph{first-order logic} (FOL) theorem prover.
In order to understand beagle's purpose and functions a basic understanding of the FOL logical system
is required. This section provides a rudimentary overview of FOL syntax and uses;
but also includes an explanation of any specialised terms and notation used throughout the paper.

\subsection{FOL basics}

\subsection{Calculi and FOL problems}

\subsection{The Superposition Calculus}

\notes{
Should contain
\begin{itemize}
\item Variables
\item Symbols
\item Predicates
\item Quantifiers
\item Notion of soundness and completeness
\item Description of a 'calculus'
\end{itemize}

\subsection{Specialised Syntax in this Paper}

\begin{itemize}
\item Positions
\end{itemize}}

\section{Automated Reasoning and Theorem Proving}
\label{sec:proving}

Automated Reasoning is a rapidly growing field of research where computer programs
are used to solve problems stated in first order logic statments or other formal logics.

Some existing theorem provers include:

\subsubsection*{SPASS}
\cite{spass}

\subsubsection*{Vampire}
\cite{vampire}

\subsubsection*{E}
\cite{eprover}

\notes{
%%%%%%%%%%%%%%%%%%%%%%%%%%%%%%%%%%
Should contain
%%%%%%%%%%%%%%%%%%%%%%%%%%%%%%%%%%
\begin{itemize}
\item Theorem prover examples
\end{itemize}
}

\section{Term Indexing}
\label{sec:indexing}

Term indexing is a technique used to better locate logical terms which match rules
in a prover's calculus.

\subsubsection*{Top Symbol Hashing}

\subsubsection*{Discriminant Trees}


\section{Fingerprint Indexing}
\label{sec:fingerprint}

\emph{Fingerprint Indexing} is a recent technique developed by \citeN{shulz12}, the creator
of the E prover.

\begin{table}[h]\begin{center}
  \caption{Fingerprint matches for Unification \protect\cite[p6]{shulz12}}
  \label{tab:unif}
  \begin{tabular}{| c || c | c | c | c | c |}
  \hline
           &  $f_1$      &  $f_2$      &  \textbf{A} &  \textbf{B} &  \textbf{N} \\ \hline \hline
  $f_1$    &  \compY &  \compN &  \compY &  \compY &  \compN \\ 
  $f_2$    &  \compN &  \compY &  \compY &  \compY &  \compN \\ 
\textbf{A} &  \compY &  \compY &  \compY &  \compY &  \compN \\
\textbf{B} &  \compY &  \compY &  \compY &  \compY &  \compY \\ 
\textbf{N} &  \compN &  \compN &  \compN &  \compY &  \compY \\ \hline
  \end{tabular}
\end{center}\end{table}

\begin{table}[h]\begin{center}
  \caption{Fingerprint matches for Matching \protect\cite[p6]{shulz12}}
  \begin{tabular}{| c || c | c | c | c | c |}
  \hline
           &  $f_1$      &  $f_2$      &  \textbf{A} &  \textbf{B} &  \textbf{N} \\ \hline \hline
  $f_1$    &  \compY &  \compN &  \compN &  \compN &  \compN \\ 
  $f_2$    &  \compN &  \compY &  \compN &  \compN &  \compN \\ 
\textbf{A} &  \compY &  \compY &  \compY &  \compN &  \compN \\
\textbf{B} &  \compY &  \compY &  \compY &  \compY &  \compY \\ 
\textbf{N} &  \compN &  \compN &  \compN &  \compN &  \compY \\ \hline
  \end{tabular}
\end{center}\end{table}

\section{The Beagle Theorem Prover}
\label{sec:beagle}
%Beagle will already be discussed in intro.
%Here we can be more technical due to background info

The core implementation of Beagle was developed by Peter Baumgartner et al. of NICTA.
Its purpose was to demonstrate the capabilities of the \emph{Weak Abstraction with Heirachic Superposition Calculus};
which allows the incorporation of prior knowledge via a `background reasoning' modules.


\subsection{The Heirachic Superposition with Weak Abstraction Calculus}

\subsection{Beagle's Shortcomings}



\section{Tools Used}

\subsection{Scala}
\label{sec:scala}

As mentioned above \beagle is written in \emph{Scala}, the Scalable Language. Scala
is a functional language and may be confusing to those who are not familiar with the
functional programming paradigm. This thesis will contain occasional snippets of
Scala code; but note that any snippets used will be accompanied by an explanation
and in general an understanding of Scala/funcitonal programming is not required.

\subsection{VisualVM}

\subsection{Eclipse}

%%% Local Variables: 
%%% mode: latex
%%% TeX-master: "thesis"
%%% End: 
