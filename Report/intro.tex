%%
%% Template intro.tex
%%

\chapter{Introduction}
\label{cha:intro}

This thesis provides a detailed account of how an optimisation technique known
as Term Indexing was added to the \beagle\ first order logic theorem prover. \Beagle\ 
is a \emph{resolution} prover which 

\section{Goals and Motivation}
\label{sec:mot}

The primary goal of this project is the implementation of Term Indexing
for the \beagle\ theorem prover. \Beagle\ implments a rather unique set of logic rules known as the \emph{\HSWAC} \cite{baum13}.
This logical calculus allows the inclusion of \emph{background reasoning} modules,
which implement a system of prior knowledge; for example the rules of integer arithmetic.

\Beagle\ was created rapidly as a demonstration and testbed for the capabilities
of its heirachic reasoning system. As such, it is lacking many key features required
to make it a viable competitor against existing theorem provers. The most significant
feature which is yet to be implemented is \emph{Term Indexing}. Term Indexing exists
to efficently locate logic expressions which are likely applicable to calculus \emph{inference rules},
used to generate new knowledge. Without this feature \beagle\ locates inference clauses
via a worst-case $O(n^)$ search; attempting each pair of clauses until a rule applies.

The specific indexing technique which will be implemented is Fingerprint Indexing.


\subsection{Measuring Performance}

After providing an implementation of Fingerprint Indexing we will wish to measure
how it has affected \beagle's performance.

\subsection{Cross Validating Fingerprint Indexing}

Fingerprint indexing itself was introduced for the first
time in \cite{shulz12}; and at the time of writing the only
implementation of Fingerprint Indexing was for Shulz's tests. By implementing this new technique in \beagle\ 
we will be providing the Term Indexing field with more evidence regarding the
overall performance of Fingerprint Indexing.

Performance results for Fingerprint Indexing in \cite{shulz12} appear very promising;
and if the \beagle\ implementation can achieve similar improvements we will
have solidifed Fingerprint Indexing as a major competitor in the field of Term Indexing.

\subsection{Improving Fingerprint Indexing}

Another side-goal of the project is to investigate and implement original
improvements to Fingerprint Indexing.
This can be achieved by \emph{tailoring} Fingerprint Indexing to \beagle\ 
and heirachic reasoning. Specifically it may be possible to take
advantage of the large number of conditions that the \HSWAC\ places on
its inference rules.

It may also be possible to find improvements to Fingerprint Indexing which
apply more generally. General improvements will be explored; but considering
that tailored improvements would result in more significant performane increases they
are considered somewhat `out of scope' for this project.

\section{Structure of this Report}
\label{sec:framework}

In this report we will...

\subsection{Background}

The background chapter (Chapter \ref{cha:background}) provides necessary background material required to understand
the concepts used in implementing Fingerprint Indexing.

This includes a literature review where we discuss 
\begin{itemize}
\item Reference glossary of important logic terms.
\item Definition of Term Indexing
\item
\end{itemize}

It will also provide
an in depth explanation of how Fingeprint Indexing works including its advantages
over other Term Indexing techniques.

\subsection{Method}

The method chapter (Chapter \ref{cha:method}) will cover in detail the 

\subsection{Results}

%%% Local Variables: 
%%% mode: latex
%%% TeX-master: "thesis"
%%% End: 
