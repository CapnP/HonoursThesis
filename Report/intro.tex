%%
%% Template intro.tex
%%

\chapter{Introduction}
\label{cha:intro}

This thesis provides a detailed account of how an optimisation technique known
as Term Indexing was added to the \beagle\ first order logic theorem prover. \Beagle\ 
is a \emph{resolution} prover which 

\section{Goals and Motivation}
\label{sec:mot}

\subsection{Current Performance of Beagle}

The primary goal of this project is the implementation of Fingerprint Indexing
for the \beagle\ theorem prover.
\Beagle\ currently lacks any decent 

\subsection{Cross Validating Fingerprint Indexing}

Fingerprint indexing itself is a very new technique; introduced for the first
time in \cite{shulz12}. At the time of writing this paper presented the only
implementation of Fingerprint Indexing. By implementing the technique in \beagle\ 
we will be providing the Term Indexing field with more evidence regarding the
overall performance of Fingerprint Indexing. 

\section{Structure of this Report}
\label{sec:framework}

In this report we will...

\subsection{Background}

The background chapter (Chapter \ref{cha:background}) provides necessary background material required to understand
the concepts used in implementing Fingerprint Indexing.

This includes a literature review where we discuss 
\begin{itemize}
\item Reference glossary of important logic terms.
\item Definition of Term Indexing
\item
\end{itemize}

It will also provide
an in depth explanation of how Fingeprint Indexing works including its advantages
over other Term Indexing techniques.

\subsection{Method}

The method chapter (Chapter \ref{cha:method})

\subsection{Results}

%%% Local Variables: 
%%% mode: latex
%%% TeX-master: "thesis"
%%% End: 
