%%
%% Template chap2.tex
%%

\chapter{Results}
\label{cha:results}

\section{Beagle Before Indexing}
\label{sec:preindexing}

We have stated several times throughout this report that the key area of improvement for \beagle
is in clause resolution. However we have yet to provide any evidence to that fact.
Here we provide some results from initial investigations (before Fingerprint Indexing
was implemented) used to identify the key areas of improvement for \beagle.

\subsection{Points of Improvement}

Refer to results for instrumentation; showing what may be improved with indexing.

\section{Indexing Metrics}
\label{sec:metrics}

\subsection{Problem Selection}
Not going to run everything against all of TPTP. Will take out a subset
of TPTP (25--50 problems) run them against this set.
Show the set and justify inclusions/exclusions

\subsection{Speed}

\subsection{False Positives}
Explain this metric and how it impacts performance

\section{Indexing Subsumption}
\label{sec:indexresults}

\subsection{Metric Results}

We observe many, even after a myriad of optimisations. Notice
that this is due to the structure of beagle; many retrieved terms
are cheaply thrown out due to other conditions (such as being a parent
term, being ordered, etc.) and do not significantly impact performance.

Refer to email from Stephan. compare false positive results


\subsection{Comparison}

\section{Indexing Simplification and Matching}
\subsection{Further Intstrumentation}

\subsection{\Beagle\ with Simplification Improvements}

\section{Tailored Improvements}
\label{sec:tailresults}

\subsection{Layer Checking}

\subsection{Metric Results}



%%% Local Variables: 
%%% mode: latex
%%% TeX-master: "thesis"
%%% End: 
