%%
%% Template chap2.tex
%%

\chapter{Results}
\label{cha:results}

\section{Beagle Before Indexing}
\label{sec:preindexing}

We have stated several times throughout this report that the key area of improvement for \beagle
is in clause resolution. However we have yet to provide any evidence to that fact.
Here we provide some results from initial investigations (before Fingerprint Indexing
was implemented) used to identify the key areas of improvement for \beagle.

\subsection{Points of Improvement}
\label{section:instr}

Refer to results for instrumentation; showing what may be improved with indexing.

\section{Indexing Metrics}
\label{sec:metrics}

\subsection{Problem Selection}
Not going to run everything against all of TPTP. Will take out a subset
of TPTP (25--50 problems) run them against this set.
Show the set and justify inclusions/exclusions

\subsection{Speed}

\subsection{False Positives}
Explain this metric and how it impacts performance.

Not as good results due to extreme other conditions on inference rules.
refer to differences in background rules. Also matching subterms!
Cheap throwaway FPs are not a concern.
totally useless even. Comment on change in how they are measured.

It is possible to na\"{\i}vley boost false positives to 0 by indexing a
huge number of positions; but this would not yield any speed improvement. Balancing false positive count with fingerprint
length is key.

\section{Indexing Subsumption}
\label{sec:indexresults}

\begin{table}[H]\begin{center}
  \caption{Totalled inference counts and indexing statistics for various versions of \beagle.}
\begin{tabular}{| l || r | r | r || r | r | r |}  \cline{2-7}
\multicolumn{1}{ c }{} & \multicolumn{3}{ |c|| }{\textbf{Inference Counts}} & \multicolumn{3}{ c| }{\textbf{Indexing Results}} \\ \cline{1-7}
Version&Sup&Demod&NegUnit&TotalFound&SupFP&SimpFP\\  \cline{1-7}
\textbf{Unmodified \footnotemark[1]}&414216&29097&1826&0&0&0\\
\textbf{Standard}&162881&41414&2452&61884768&15525&39778148\\
\textbf{Enhanced}&162997&41435&2454&58535681&15401&39779224\\ \hline
\end{tabular}\end{center}\end{table}

\begin{table}[H]\begin{center}
  \caption{Totalled timing results for various versions of \beagle.}
\begin{tabular}{| l || r | r | r | r | r | r |}  \cline{2-7}
\multicolumn{1}{ c }{} & \multicolumn{6}{| c| }{\textbf{Time Spent (seconds)}} \\ \cline{1-7}
Version&Indexing&Retrieving&Sup&Demod&NegUnit&Total\\  \cline{1-7}
\textbf{Unmodified \footnotemark[1]}&0&0&730.44&9.44&31.99&5623.21\\
\textbf{Standard}&28.4&38.73&254.17&41.66&3.18&381.36\\
\textbf{Enhanced}&22.91&20.29&180.54&32.6&2.51&281.38\\ \hline
\end{tabular}\end{center}\end{table}

\footnotetext[1]{This version failed to solve two problems within 8 hours (28800 seconds).
These results are not included in the totals. See verbose table of results (Section \ref{sec:unres})}

\subsection{Metric Results}
\label{section:instr2}

We observe many, even after a myriad of optimisations. Notice
that this is due to the structure of beagle; many retrieved terms
are cheaply thrown out due to other conditions (such as being a parent
term, being ordered, etc.) and do not significantly impact performance.

Refer to email from Stephan. compare false positive results

\subsection{Indexing Simplification and Matching}

\subsection{Tailored Improvements}
\label{sec:tailresults}


\section{Comparing Various Fingerprint Sampling Positions}


\begin{table}[H]\begin{center}
  \caption{Totalled inference counts and indexing statistics for various Fingerprint sampling sets.}
\begin{tabular}{| l || r | r | r || r | r | r |}  \cline{2-7}
\multicolumn{1}{ c }{} & \multicolumn{3}{ |c|| }{\textbf{Inference Counts}} & \multicolumn{3}{ c| }{\textbf{Indexing Results}} \\ \cline{1-7}
Sample Set&Sup&Demod&NegUnit&TotalFound&SupFP&SimpFP\\  \cline{1-7}
\textbf{FP3W}&164574&42402&2473&59433145&72051&40037948\\ 
\textbf{FP4M}&150154&35709&1964&58989318&29469&40073471\\
\textbf{FP6M}&146861&35326&1960&58119897&17641&39916687\\ 
\textbf{FP7}&161411&41005&2441&58530669&23903&39818531\\
\textbf{FP8X2}&161741&40876&2439&58336597&11754&39823989\\ \hline 
\end{tabular}\end{center}\end{table}

\begin{table}[H]\begin{center}
  \caption{Totalled timing results for various Fingerprint sampling sets.}
\begin{tabular}{| l || r | r | r | r | r | r |}  \cline{2-7}
\multicolumn{1}{ c }{} & \multicolumn{6}{| c| }{\textbf{Time Spent (seconds)}} \\ \cline{1-7}
Sample Set&Indexing&Retrieving&Sup&Demod&NegUnit&Total\\  \cline{1-7}
\textbf{FP3W}&12.48&14.86&177.82&28.98&1.79&265.65\\ 
\textbf{FP4M}&14.37&15.02&173.32&31.53&1.83&261.84\\
\textbf{FP6M}&18.74&17.58&168.79&30.56&2.12&259.02\\ 
\textbf{FP7}&22.26&19.82&180.13&35.23&2.38&282.22\\
\textbf{FP8X2}&51.73&34.23&195.18&42.75&4.07&331.01\\ \hline 
\end{tabular}\end{center}\end{table}

%%% Local Variables: 
%%% mode: latex
%%% TeX-master: "thesis"
%%% End: 
