%%
%% Template conclusion.tex
%%

\chapter{Conclusion}
\label{cha:conclusion}

\section{Implementing Fingerprint Indexing for \Beagle}

This report has covered in detail how Term Indexing was added to \beagle;
involving initial research on Term Indexing, constructing the index, instating it
to active use, adding optimisations and finally instrumenting and measuring performance of the final implementation.

Our initial research included selecting Fingerprint Indexing as our desired indexing technique.
Fingerprint Indexing was chosen for its relative simplicity over other indexing techniques
and for being a new technique which has not reached its full potential. Other
research conducted prior to implementation included gaining a thorough understanding
of existing \beagle\ code and the logical calculus it implements: the \HSWAC.

Constructing our Fingerprint Index required a complex Fingerprint Index data structure
along with the ability to insert and retrieve terms. Our prior research indicated that
our index would need to be capable of indexing all possible \emph{subterms}; requiring
some caution that identical subterms were not overwritten.

Actually making use of our index required referring again to our earlier research in order to
find what areas of \beagle\ may be improved with indexing. Two main blocks
of code were found applicable; the superposition inference rule and simplification
of clauses. Indexing superposition required significant re-structuring of \beagle's
inference code and separate implementations of the \emph{from} and \emph{into} cases.
Indexing simplification required adding two new Fingerprint Indices with different
configurations; built exclusively to index unit clauses.

This paper also discussed some original optimisations made to Fingerprint Indexing.
These optimisations worked by taking advantage of restrictions in the \HSWAC\ and
its natural segregation of terms into foreground and background parts.

After completing the implementation and optimisation of Fingerprint Indexing its performance
was evaluated against a carefully selected set of logic problems. Performance over this
set of problems was used to compare a variety of indexed versions of \beagle\ with each other and the original unindexed version. 
Section \ref{sec:why} will recap the results of performance testing and what
has been achieved by implementing Fingerprint Indexing. However, prior to this final statement,
we will discuss some alternative optimisation ideas conceived during the course of the project;
which did not make it to the implementation stage.

\section{Future Improvements}
\label{sec:future}

Here we list some thoughts on possible improvements to \beagle\ and Fingerprint
indexing in general; which were either not relevant to the current work or
were not thoroughly investigated due to time constraints.


\subsection{Inverted Data Structure}
\label{sec:refd}

Terms are only \emph{added} to the Fingerprint Index when we infer new information;
but index \emph{retrieval} is attempted constantly at a high frequency. It is possible
that by moving computation from index retrieval over to index addition some time could
be saved.

This could be achieved by having each term maintain a its own set of terms which
are compatible with it. Keeping these sets up-to-date would be very difficult; as we would
need to locate the compatible terms of any newly inferred ones. This would require a standard
Fingerprint Index on top of the reversed one. When we consider that indexing superposition
would also need to take subterms into account, this technique would require
an exceptionally large memory overhead and term addition would become exceedingly difficult.
Thus it is unlikely that this idea merits any real performance improvement; except
perhaps in very particular cases.

\subsection{More Fingerprint Indices}

When we applied Fingerprint Indexing to simplification in Section \ref{sec:simp}
we created two new indices for unit clauses rather than re-using the superposition
index; which will generally contain thousands of terms useless to simplification.

It is possible that more performance could be squeezed out of creating more Fingerprint
Indices to handle other special cases. Take the Equality Factoring rule for instance
(See Section \ref{sec:calc}). We dismissed the notion of applying indexing to this rule
since it only needs to look for unifiable terms within a single clause. It is possible
that the performance of Equality Factoring could be improved by giving each clause
its own `mini-index' which indexes only its own literals. Unfortunately this technique
could only hope to improve performance in the case of \emph{enormous} individual clauses.

\subsection{Retrieval Caching}
In the case that many retrievals for the same query term are observed it may increase
performance to \emph{cache} the compatible set of terms so that they may be retrieved
instantly the next time the term is queried. Essentially this is a scaled back version
of the over-ambitious index reversion proposed in Section \ref{sec:refd}.

Retrieval caching could potentially be used to significant effect for superposition
indexing; in particular for the \emph{into case} described in Section \ref{ref:indsup}.
In this case we must loop over each subterm, meaning that we repeatedly query the index
for trivial bottom-level terms; such as a single function symbol or variable. These cases
could be cached and retrieved instantly.

Implementing this cache would require great care as any newly indexed terms must
also be added to any matching cache sets. This puts a restriction on how many queries
can be cached since each extra one slows down the process of adding to the index.
Alternatively, cache sets could be discarded when a new term is added; reducing their
usefulness but ensuring correctness.

\subsection{Dynamic Position Sets}
We discovered in the results for varied position sampling sets (Section \ref{sec:fingcomp}) that
small fingerprints can perform exceedingly well even when observing a tremendous
amount of false positives.
Thus it may be possible to increase performance by automatically tailoring the position
set to the current problem (during \beagle's initial setup). For example, if we know
that most terms generated will be distinguishable based on the top symbol alone
then we would probably achieve peak performance by indexing only a couple of positions.
The problem here is that we could never predict what terms generated during the inference process will look like.

Consider then the possibility of \emph{Dynamic Position Sets}, where we start \beagle\ off with a
very large set of positions (say FP8X2) and remove positions over time if they
are determined to be hindering performance. This would require a more complex Fingerprint
Index tree which is capable of performing this removal operation within a reasonable
time frame. Note that \emph{adding} positions would be much more difficult than removing them, since
we would need to re-sample all terms in the index.

\subsection{More Thorough Tests and Comparisons}

Unfortunately due to the time and resources available it was not viable to run
all versions of \beagle\ over the full collection of TPTP problems. This would
be worthwhile since a more in depth comparison 
of \beagle\ versions would better solidify the performance increase given by Fingerprint Indexing.
However, since we were careful to be \emph{fair}
while selecting problems it is unlikely that observed results would stray far
from our conclusions.

It would also be interesting to compare the performance of Fingerprint Indexing
to other techniques. Unfortunately, comparing against indexing implementations in theorem provers other
than \beagle\ would not provide relevant results, due to \beagle's rather
exceptional calculus. Thus comparing various techniques would require a full
implementation of each one within \beagle, a task which is obviously outside the scope of the current
project. 

\section{The Benefits of Term Indexing for the \Beagle\ Theorem Prover }
\label{sec:why}

With \beagle\ now fully augmented with Fingerprint Indexing its performance
has improved significantly. The standard Fingerprint Indexing implementation was
able to improve performance for large problems; allowing some
problems which where previously unsolvable in 8 hours be solved in under a minute.
Our results have shown that adding calculus specific tailored
enhancements further improves performance; decreasing the runtime of each superposition inference by 35\%
on average and up to 50\% for large problems (not including the problems which unindexed \beagle\ could not solve,
where an improvement of at least 90\% was observed). This enhanced version
was also able to successfully improve performance for many small and simple problems,
where indexing should have only served to overcomplicate the inference process.

By comparing a variety of Fingerprint Index configurations we were able to
identify the optimal range for fingerprint lengths. These results matched well
with the results by Stephan \citeN{shulz12}, cross validating and confirming both.
We were also able to provide a more in depth analysis of Fingerprint Indexing performance due
to considering of a variety of performance metrics; rather than just total runtime.
This analysis was able to solidify and confirm Shulz's proposition that Fingerprint
Indexing is viable to compete with any other current Term Indexing technique.

By adding Term Indexing to the \beagle\ theorem prover we have significantly increased
its viability as a competitor in the field. The \HSWAC\ may now be applied to a range
of problems which were previously intractable; allowing the continued use of \beagle\ for
the research of automated hierarchic reasoning.

%%% Local Variables: 
%%% mode: latex
%%% TeX-master: "thesis"
%%% End: 
