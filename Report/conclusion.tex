%%
%% Template conclusion.tex
%%

\chapter{Conclusion}
\label{cha:conclusion}

\section{The Benefits of Term Indexing for the \Beagle\ Theorem Prover }
\label{sec:why}

\section{Future Improvements}
\label{sec:future}

Here we list some thoughts on possible improvements to \beagle\ and Fingerprint
indexing in general; which were either not relevant to the current work or
were not investigated due to time constraints.

\subsection{More Thorough Testing}

\subsection{Extended Data Structures}
The focus of this thesis has been with high-level logical improvements. Here
we present some thoughts on how 'low-level' speed could be saved at the cost of memory.

Drop the tree structure. Have an entry for each possible fingerprint and add terms
to each bin they are compatible with during indexing. Moves runtime from
retrieval (called VERY often) to indexing (called barely ever, comparatively) but requires a
tremendous amount of memory.

\subsection{More Fingerprint Indices}

\subsection{Extensions to Fingerprint Indexing}

\subsubsection{Symbol count / Other Features}
\subsubsection{Retrieval Caching}

I believe retrieval caching could be used to significant effect for superposition
indexing; in particular for the \emph{into case} described in section \ref{ref:indsup}.
In this case we must loop over each subterm, meaning that we repeatedly query the index
for trivial bottom-level terms like a single function symbol or variable. These cases
could be cached and retrieved instantly.

Implementing this cache would require great care as any newly indexed terms must
also be added to any matching cache sets. This puts a restriction on how many queries
can be cached since each extra one slows down the process of adding to the index.

\subsubsection{Dynamic Fingerprinting}
Negligible due to length/FP balance. Can make static ones arbitrarily good.

Tailoring to a problem is good though, consider example where many \textbf{B}sare generated,
 but the position cannot be reached due to a maximum function arity.

\subsection{Comparing Indexing Techniques}

\section{Final Thoughts}

By considering more indepth results we were able to better rate the performance
of Fingerprint Indexing rather than just rate it on runtime.

%%% Local Variables: 
%%% mode: latex
%%% TeX-master: "thesis"
%%% End: 
